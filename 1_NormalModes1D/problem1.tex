
\documentclass{article}

\usepackage{ amssymb }
\usepackage{ mathrsfs }
\usepackage{graphicx} % Required for the inclusion of images

\setlength\parindent{0pt} % Removes all indentation from paragraphs

%\renewcommand{\labelenumi}{\alph{enumi}.} % Make numbering in the enumerate environment by letter rather than number (e.g. section 6)

%----------------------------------------------------------------------------------------
%	INFORMACION DEL DOCUMENTO
%----------------------------------------------------------------------------------------


\begin{document}
\title{Physics Problems} % Title



\author{ For Pecouzaustralianovsky} % Author name

\date{\today}
\maketitle







\section{Normal Modes in 1D}

\textbf{a. Discrete}\\

Consider a system of N particles which is subject to a restoring force proportional to the separation to the equilibrium position ( $ F \alpha x $ ). How many normal modes has this system?\\

\textbf{b. Continuous }\\

Consider now a continuous system of particles confined to a length $L$ and fixed borders. Which are normal wave lengths ($\lambda_n$), the normal frequencies  ($\nu_n$) in terms of $\nu$(the speed of waves in the medium) and is it correct to think that this system has infinite normal modes. If the it does not have infinite normal modes, give a order of magnitude of the number of normal modes.
%\hline

\vspace{2mm}

\begin{center}
         \noindent\rule{8cm}{0.4pt}\\

\end{center}

\noindent [1] A. P. French. Vibrations and Waves. 2003

\end{document}